% ______________________________________________________________________________
%
%   1DV600 - Software Technology
%   Assignment 2 -- "Analysis, Design and Implementation"
%
%  Author:  Jonas Sjöberg
%           Linnaeus University
%           js224eh@student.lnu.se
%           https://github.com/jonasjberg
%
%    Date:  2017-02-16 -- 2017-02-19
%
% License:  Creative Commons Attribution 4.0 International (CC BY 4.0)
%           <http://creativecommons.org/licenses/by/4.0/legalcode>
%           See LICENSE.md for additional licensing information.
% ______________________________________________________________________________


% ______________________________________________________________________________
\section{Task 1 -- Analysis}

% ______________________________________________________________________________
\subsection{Subtask A -- Identifying Use Cases}\label{task-1a}
\paragraph{Instructions}\label{task-1a-instructions}
from the course Wiki\cite{1dv600:lab2:instructions}:

\begin{quote}
  The UML way of documenting requirements is through use cases and in this
  first task you are to identify and document the use cases used in the system.
  When documenting a use case, it is customary to specify initiation, pre and
  post conditions, the primary flow as well as some additional ones.  Do this
  for two of your use cases and describe the flows using activity diagrams.
  Also write down your own reflections on identifying use cases and documenting
  them in about 100 words.
\end{quote}


\subsubsection{Use case 1 -- Scenario}\label{task-1a-usecase1scenario}
%
% TODO: Specify: Initiation
%                Pre- and post-conditions
%                The primary flow
%                Some additional ones

% TODO: Describe the flows using activity diagrams.

The first use case is based on the following scenario:

The user Gibson visits the book library application website to add a newly
purchased book to his collection. Gibson first clicks the GUI button ``New
Book'' which brings up another page with a number of empty metadata fields.
Gibson then opens his physical book and fills out the fields to the best of his
ability. He does this by manually clicking with the mouse in each text field,
looks up the data in his book and fills out the textfield with the data.  After
Gibson has filled out all the metadata fields, he clicks the ``Save'' button
which brings him back to the page with the main view.
The entered data is validated by the application. If the data is correct, it is
handed over to the backend storage database.

\subsubsection{Use case 1 -- Use Case Specification}\label{task-1a-usecase1}

\includepdf[pages=-,
            pagecommand={},
            width=\textwidth]{use-case-1.pdf}


\subsubsection{Use case 2}\label{task-1a-usecase2}
%
% TODO: Specify: Initiation
%                Pre- and post-conditions
%                The primary flow
%                Some additional ones

% TODO: Describe the flows using activity diagrams.


\subsubsection{Reflect}\label{task-1a-reflect}
%
% TODO: Write down your own reflections in about 100 words, on:
%         * Identifying use cases
%         * Documenting use cases



% ______________________________________________________________________________
\subsection{Subtask B -- Robustness Diagram}\label{task-1b}
\paragraph{Instructions}\label{task-1b-instructions}
from the course Wiki\cite{1dv600:lab2:instructions}:

\begin{quote}
  A non­standardised, but yet highly valuable diagram, in UML is the robustness
  diagram in which in a way is a simplification of communication and
  collaboration diagrams. They are used to analyse the steps of a use case and
  validate the business logic for them, that the use cases are sufficiently
  robust to represent the usage requirements for the system.  Create robustness
  diagrams for the use cases you identified in task a. In your reflection
  document you write down about 100 words on your experience using robustness
  diagrams.
\end{quote}


\subsubsection{Robustness diagram for use case 1}\label{task-1b-robust1}
%
% TODO: Create robustness diagrams for the use cases you identified in task A. 


\subsubsection{Robustness diagram for use case 2}\label{task-1b-robust2}
%
% TODO: Create robustness diagrams for the use cases you identified in task A. 


\subsubsection{Reflect}\label{task-1b-reflect}
%
% TODO: Write about 100 words on your experience using robustness diagrams.



% ______________________________________________________________________________
\subsection{Subtask C -- Use Case Realization}\label{task-1c}
\paragraph{Instructions}\label{task-1c-instructions}
from the course Wiki\cite{1dv600:lab2:instructions}:

\begin{quote}
  To specify more in detail what a use case is supposed to do, i.e. to realize
  it, it is common to use sequence diagrams. In the previous assignment you
  implemented the use case "List Books" and in this subtask you are to show a
  use case realization of that in the form of a sequence diagram. In addition
  to that, do the same for the use case "Delete Book.  Again, write down your
  reflections on use case realisations in about 100 words.
\end{quote}


\subsubsection{"List Books" sequence diagram}\label{task-1c-sequence1}
%
% TODO: Use case realizations with sequence diagrams for this case.


\subsubsection{"Delete Book" sequence diagram}\label{task-1c-sequence2}
%
% TODO: Use case realizations with sequence diagrams for this case.


\subsubsection{Reflections}\label{task-1c-reflect}
%
% TODO: Write about 100 words of reflections on use case realisations.

